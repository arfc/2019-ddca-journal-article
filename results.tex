\section{Results}
To demonstrate \deploy's capability to effectively conduct transition
scenario analysis and meet the objectives described in section 
\ref{sec:obj}, this section will (1) demonstrate 
\deploy's capability in simple transition scenarios, (2) set up 
EG01-EG23, EG01-EG24, EG01-EG29, and EG01-EG30 transition scenarios
and compare the results of using each prediction methods for each 
scenario. 
The input files and scripts to produce the results and plots in this
paper can be reproduced using \cite{d3ploy_doi_2019}. 

\subsection{Demonstration of \deploy's capabilities}
\label{sec:demo}
Constant, linearly increasing, and sinusoidal power demand simulations 
are conducted to demonstrate \deploy's capabilities for 
simulating transition scenarios and to inform decisions about 
input parameters when setting up larger transition scenarios 
with many facilities.  
These simulations are basic transition scenarios that only include 
three types of facilities: \texttt{source}, \texttt{reactor}, and 
\texttt{sink}. 
All simulations have ten initial \texttt{reactor} facilities 
(\texttt{reactor1} to \texttt{reactor10}). 
These reactors have staggered cycle lengths and lifetimes to prevent 
simultaneous refueling and set up gradual decommissioning. 
\Deploy is set up to deploy \texttt{new reactor} facilities
to meet the loss of power supply introduced from the decommissioning 
of the initial \texttt{reactor} facilities. 
The \deploy input parameters for each simulation is shown in Table 
\ref{tab:demonstrations}. 
Figure \ref{fig:powerplots} shows the user-defined power demand curves 
that \deploy needs to deploy facilities meet for each simulation.

\begin{table}[H]
    \resizebox{\textwidth}{!}{%
    \begin{tabular}{|l|l|c|l|l|}
    \hline
    \multirow{2}{*}{}                         & \multicolumn{1}{c|}{\multirow{2}{*}{\textbf{Input Parameter}}} & \multicolumn{3}{c|}{\textbf{Simulation Description}}                                                                                                                                                                                                                                                       \\ \cline{3-5} 
                                              & \multicolumn{1}{c|}{}                                          & \multicolumn{1}{l|}{\textbf{Constant Power}}                                                                 & \textbf{\begin{tabular}[c]{@{}l@{}}Linearly Increasing \\ Power\end{tabular}}                  & \textbf{Sinusoidal Power}                                                                  \\ \hline
    \multirow{5}{*}{\textbf{Required Inputs}} & Demand driving commodity                                       & \multicolumn{3}{c|}{Power}                                                                                                                                                                                                                                                                                 \\ \cline{2-5} 
                                              & Demand equation                                                & \multicolumn{1}{l|}{10000 MW}                                                                                & \begin{tabular}[c]{@{}l@{}}t\textless 40: 10000 MW\\ t\textgreater{}=40: 250*t MW\end{tabular} & 1000*$\sin(\pi*t/3)$+10000                                                                 \\ \cline{2-5} 
                                              & Facilities it controls                                         & \multicolumn{3}{c|}{Source, reactor, sink}                                                                                                                                                                                                                                                                 \\ \cline{2-5} 
                                              & Prediction method                                              & \multicolumn{1}{l|}{\begin{tabular}[c]{@{}l@{}}Power: FFT\\ Fuel: MA\\ Spent fuel: MA\end{tabular}}          & \begin{tabular}[c]{@{}l@{}}Power: FFT\\ Fuel: MA\\ Spent fuel: FFT\end{tabular}                & \begin{tabular}[c]{@{}l@{}}Power: HW\\ Fuel: MA\\ Spent fuel: FFT\end{tabular}             \\ \cline{2-5} 
                                              & Deployment Driving Method                                      & \multicolumn{3}{c|}{Installed Capacity}                                                                                                                                                                                                                                                                    \\ \hline
    \multirow{2}{*}{\textbf{Optional Inputs}} & Buffer type                                                    & \multicolumn{3}{c|}{Absolute}                                                                                                                                                                                                                                                                              \\ \cline{2-5} 
                                              & Buffer size                                                    & \multicolumn{1}{l|}{\begin{tabular}[c]{@{}l@{}}Power: 3000 MW\\ Fuel: 0 kg \\ Spent fuel: 0 kg\end{tabular}} & \begin{tabular}[c]{@{}l@{}}Power: 2000 MW\\ Fuel: 1000 kg \\ Spent fuel: 0 kg\end{tabular}     & \begin{tabular}[c]{@{}l@{}}Power: 2000 MW\\ Fuel: 1000 kg \\ Spent fuel: 0 kg\end{tabular} \\ \hline
    \end{tabular}%
    }
    \caption{\Deploy's input parameters for the basic transition scenarios.}
    \label{tab:demonstrations}
    \end{table}

    \begin{figure}[H]
        \begin{center}
            \includegraphics[scale=0.37]{./figures/powerplots.png}
        \end{center}
            \caption{Power demand curves for basic transition scenarios.}
        \label{fig:powerplots}
    \end{figure}

    \begin{figure}[H]
        \centering
        \begin{subfigure}[t]{\textwidth}
        \centering
            \includegraphics[width=0.8\linewidth]{figures/constanttransition-power.png} 
            \caption{The power demand is a user-defined equation and power is supplied by the reactors.}
            \label{fig:constanttransition-power}
        \end{subfigure}
        \begin{subfigure}[t]{0.65\textwidth}
            \centering
            \includegraphics[width=\linewidth]{figures/constanttransition-fuel.png} 
            \caption{Fuel is demanded by reactors and supplied by source facilities.}
            \label{fig:constanttransition-fuel}
        \end{subfigure}
        \begin{subfigure}[t]{0.65\textwidth}
            \centering
            \includegraphics[width=\linewidth]{figures/constanttransition-spentfuel.png} 
            \caption{Spent Fuel is supplied by reactors and the capacity is provided by sink facilities.}
            \label{fig:constanttransition-spentfuel}
        \end{subfigure}
        \caption{Transition Scenario: Constant Power Demand of 10000MW}
    \end{figure}

The undersupply results of these simulations is shown in Table 
\ref{tab:transition-scenario-results}. 
\begin{table}[H]
    \centering
    \caption {Undersupply results for each commodity in each scenario}
	\label{tab:transition-scenario-results}
    \begin{tabular}{|l|l|p{4cm}|}
    \hline
    \textbf{Basic Transition Scenario}    & \textbf{Commodity}    & \textbf{No. of time steps with undersupply} \\ \hline
    \multirow{2}{*}{\textbf{Constant Power}} & Fuel & 1 \\ \cline{2-3}
                                             & Power & 0 \\ \cline{2-3}
                                             & Spent Fuel & 0 \\ \hline
    \multirow{2}{*}{\textbf{Linearly Increasing Power}} & Fuel & 1 \\ \cline{2-3}
                                             & Power & 0 \\ \cline{2-3}
                                             & Spent Fuel & 0 \\ \hline
    \multirow{2}{*}{\textbf{Sinusoidal Power}} & Fuel & 1 \\ \cline{2-3}
                                             & Power & 1 \\ \cline{2-3}
                                             & Spent Fuel & 0 \\ \hline
    \end{tabular}
\end{table}

\subsubsection{Basic Transition Scenario: Constant Demand}

\subsection{Comparison of Prediction Methods}