\section{Methodology}
%Description of D3ploy
In a \Cyclus \gls{NFC} simulation, at every time step, \deploy 
predicts demand and supply of each commodity for the next time 
step. 
If there is an undersupply of any commodity based 
on the predicted values, \deploy deploys the fewest number 
of available facilities to meet the predicted demand 
with minimal oversupply.  
Figure \ref{fig:flow} shows the logic flow of \deploy 
at every time step. 

\begin{figure}[H]
	\centering
    \begin{tikzpicture}[node distance=2.5cm]
    \tikzstyle{every node}=[font=\large]
	\node (Start) [bblock] {\textbf{Start of time step ($t$).}};
	\node (Predict) [bblock, below of=Start] {\textbf{Calculate \\ $D_p(t+1)$ and $S_p(t+1)$ for a commodity}};
	\node (IsThere) [oblock, below of=Predict]{\textbf{$U(t+1) = S_p(t+1)-D_p(t+1)$}};
	\node (Deploy) [bblock, below of=IsThere, xshift = -3.5cm]{\textbf{Deploy}};
    \node (NoDeploy) [bblock, right of=Deploy, xshift = 3.5cm]{\textbf{No Deploy} };
    \node (All) [oblock, below of=Deploy, xshift = 3.5cm] {\textbf{Is this done for all commodities?}};
    \node (End) [bblock, below of=All] {\textbf{Proceed to next time step.}};
	
	\draw [arrow] (Start) -- (Predict); 
	\draw [arrow] (Predict) -- (IsThere);
    \draw [arrow] (IsThere) -- node[anchor=east] {$U(t+1) <$ buffer} (Deploy);
    \draw [arrow] (IsThere) -- node[anchor=west] {$U(t+1) \geq$ buffer} (NoDeploy);
    \draw [arrow] (Deploy) -- (All);
    \draw [arrow] (NoDeploy) -- (All);
    \draw [arrow] (All) -- node[anchor=west] {yes} (End);
    \draw [arrow] (All) -- ([shift={(-3.9cm,0.7cm)}]All.south west)-- node[anchor=east] {no} ([shift={(-3.9cm,-1cm)}]Predict.north west)--(Predict);
    \draw [arrow] (End) |-([shift={(3cm,-0.5cm)}]End.south east)-- ([shift={(3cm,0.5cm)}]Start.north east)-|(Start);
    \end{tikzpicture}
    \label{fig:flow}
    \caption{\Deploy logic flow at every time step in \Cyclus \cite{chee_demonstration_2019}.}
\end{figure}

\Deploy's overall objective is to minimize the number of time 
steps where there is a undersupply of any commodity. 
However, the ultimate objective is to ensure that there is no 
undersupply of power. 
One of the key issues that \gls{NFCSim}s face is that despite
sufficient installed reactor capacity to meet the power 
demand, there is insufficient supply of fabricated/reprocessed 
fuel at certain time steps, resulting in idle capacity.  


\subsection{Structure}
%Description of front end and back end of fuel cycle 
%Demand Driven vs. Supply Driven 

\subsection{Basic Input Variables}
\subsection{Prediction Algorithms}
\subsection{}