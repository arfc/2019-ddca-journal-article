\section{Methodology}
%Description of D3ploy
In a \Cyclus \gls{NFC} simulation, at every time step, \deploy 
predicts demand and supply of each commodity for the next time 
step. 
If there is an undersupply of any commodity based 
on the predicted values, \deploy deploys the fewest number 
of available facilities to meet the predicted demand.  
Figure \ref{fig:flow} shows the logic flow of \deploy 
at every time step. 

\begin{figure}[H]
	\centering
    \begin{tikzpicture}[node distance=2.5cm]
    \tikzstyle{every node}=[font=\large]
	\node (Start) [bblock] {\textbf{Start of time step ($t$).}};
	\node (Predict) [bblock, below of=Start] {\textbf{Calculate \\ $D_p(t+1)$ and $S_p(t+1)$ for a commodity}};
	\node (IsThere) [oblock, below of=Predict]{\textbf{$U(t+1) = S_p(t+1)-D_p(t+1)$}};
	\node (Deploy) [bblock, below of=IsThere, xshift = -3.5cm]{\textbf{Deployment of facility}};
    \node (NoDeploy) [bblock, right of=Deploy, xshift = 3.5cm]{\textbf{No Deployment} };
    \node (All) [oblock, below of=Deploy, xshift = 3.5cm] {\textbf{Is this done for all commodities?}};
    \node (End) [bblock, below of=All] {\textbf{Proceed to next time step.}};
	
	\draw [arrow] (Start) -- (Predict); 
	\draw [arrow] (Predict) -- (IsThere);
    \draw [arrow] (IsThere) -- node[anchor=east] {$U(t+1) <$ buffer} (Deploy);
    \draw [arrow] (IsThere) -- node[anchor=west] {$U(t+1) \geq$ buffer} (NoDeploy);
    \draw [arrow] (Deploy) -- (All);
    \draw [arrow] (NoDeploy) -- (All);
    \draw [arrow] (All) -- node[anchor=west] {yes} (End);
    \draw [arrow] (All) -- ([shift={(-3.9cm,0.7cm)}]All.south west)-- node[anchor=east] {no} ([shift={(-3.9cm,-1cm)}]Predict.north west)--(Predict);
    \draw [arrow] (End) |-([shift={(3cm,-0.5cm)}]End.south east)-- ([shift={(3cm,0.5cm)}]Start.north east)-|(Start);
    \end{tikzpicture}
    \label{fig:flow}
    \caption{\Deploy logic flow at every time step in \Cyclus \cite{chee_demonstration_2019}.}
\end{figure}

\Deploy's overall objective is to ensure that there is no 
undersupply of power. 
The sub-objectives are : (1) to minimize the number of time 
steps of undersupply or under capacity of any 
commodity, (2): to minimize excessive oversupply of all commodities.
This is a reflection of reality in which it is important to 
never have an undersupply of power on the grid by ensuring power 
plants are never undersupplied of fuel, while not 
having excessive over supply resulting in a burden to store unused 
supplies. 
One of the key issues that \gls{NFCSim}s face is that despite
sufficient installed reactor capacity to meet the power 
demand, there is insufficient supply of fabricated/reprocessed 
fuel at certain time steps, resulting in idle capacity.  

\subsection{Structure}
%Description of front end and back end of fuel cycle 
%Demand Driven vs. Supply Driven 
In \deploy, two different institutions were implemented for 
front-end and back-end fuel cycle facilities: 
\texttt{DemandDrivenDeploymentInst} and 
\texttt{SupplyDriven} 
\noindent
\texttt{DeploymentInst} respectively. 
This distinction was made because front-end facilities 
are deployed to meet demand for the commodity they produce. 
Whereas, back-end facility are deployed to meet supply for the 
commodity they provide capacity for. 
For example, for front end facilities, a reactor facility 
demands fuel and \texttt{DemandDrivenDeploymentInst} 
triggers deployment of fuel fabrication facilities to create 
supply meeting demand for fuel to prevent undersupply. 
For back end facilities, the reactor generates spent fuel and 
\texttt{SupplyDrivenDeploymentInst} triggers deployment of 
waste storage facilities to create capacity meeting the supply 
of spent fuel to prevent under capacity. 

\subsection{Input Variables}
Table \ref{tab:inputs} describes the input variables that a user 
uses to customize their simulation. 
Essentially, the user must define the facilities controlled by 
\deploy, their capacities, the simulation driving commodity, 
its demand equation, deployment driving method, and which method
predicts supply and demand. 
The user also has the option to define supply/capacity buffers for 
each commodity, facility preferences, and facility constraints. 

In-depth descriptions of each input variable are found in the 
subsequent sections. 


\begin{table}[H]
	\resizebox{\textwidth}{!}{%
	\begin{tabular}{|l|l|p{7cm}|}
	\hline
											  & \textbf{Input Parameter}                                                           & \textbf{Examples}                                                                                                          \\ \hline
	\multirow{5}{*}{\textbf{Required}} & Demand driving commodity                                                           & Power, Fuel, Plutonium, etc.                                                                                                                      \\ \cline{2-3} 
											  & Demand equation                                                                    & P(t) = 10000, sin(t), 10000*t                                                                                                                 \\ \cline{2-3} 
											  & Facilities it controls                                                             & Fuel Fab, LWR reactor, SFR reactor, Waste repository, etc.                                                                                                      \\ \cline{2-3} 
											  & Capacities of the facilities                                                       & 3000 kg, 1000 MW, 50000 kg                                                                                                     \\ \cline{2-3} 
											  & Prediction algorithm                                                                  & \begin{tabular}[c]{@{}l@{}}Power: fast fourier transform\\ Fuel: moving average\\ Spent fuel: moving average\end{tabular} \\ \cline{2-3} 
											  & Deployment driven by & Installed Capacity/Supply                                                                                                                    \\ \hline
	\multirow{4}{*}{\textbf{Optional}} & Supply/Capacity Buffer type                                                                        & Absolute                                                                                                                  \\ \cline{2-3} 
											  & Supply/Capacity Buffer size                                                                        & \begin{tabular}[c]{@{}l@{}}Power: 3000 MW\\ Fuel: 0 kg \\ Spent fuel: 0 kg\end{tabular}                                   \\ \cline{2-3} 
											  & Facility preferences                                                               & \begin{tabular}[c]{@{}l@{}}LWR reactor = 100-t\\ SFR reactor = t-100 \end{tabular}          \\ \cline{2-3} 
											  & Facility constraint                                                              & SFR reactor constraint = 5000kg of Pu            \\ \hline	
			
											\end{tabular}%
	}
	\caption{\Deploy's required and optional input parameters with examples.}
	\label{tab:inputs}
	\end{table}

\subsubsection{Deployment Driving Method}
The user has the choice of deploying facilities based on difference 
between predicted demand and supply, or predicted demand and 
installed capacity. 
There are two advantages of using installed capacity over predicted 
supply. 
First, to prevent over deployment of facilities that have an
intermittent supply. 
For example, reactor facilities have a designated refueling time. 
A user might not want \deploy to deploy more reactor facilities 
to make up for the lack of power supply caused by a gap in 
supply during refueling. 
Second, to prevent infinite deployment of a facility that provides 
a commodity that is no longer available in the simulation. 
For example, in a transition scenario from \gls{LWR}s to \gls{SFR}s, 
the reprocessing plant that fabricates \gls{SFR} fuel might demand 
for Pu after the inventory accumulated by \gls{LWR}s was used up 
and there are no more \gls{LWR} facilities to generate Pu. 
This will result in \deploy deploying infinite reprocessing 
facilities to generate \gls{SFR} fuel despite the lack of Pu 
to generate the fuel. 
This can be avoided by constraining \gls{SFR} deployment until a 
sizable inventory of Pu is accumulated in the simulation. 

\subsubsection{Supply/Capacity Buffer}
In \texttt{DemandDrivenDeploymentInst}, the user has the option 
to provide a supply buffer to specific commodities so that 
\deploy will deploy facilities to meet predicted demand and the
additional buffer. 
In \texttt{SupplyDrivenDeploymentInst}, the user has the option 
to provide a capacity buffer to specific commodities so that 
\deploy will deploy facilities to meet predicted supply and the
additional buffer.
The buffer can be defined as a percentage (equation \ref{eq:perc}) 
or absolute value (equation \ref{eq:abs}). 

\begin{equation}
    \label{eq:perc}
    S_{pwb} = S_{p}*(1+d)
\end{equation}
\begin{equation}
    \label{eq:abs}
    S_{pwb} = S_{p}+a
\end{equation}
where $S_{pwb}$ is predicted supply/capacity with buffer, 
$S_p$ is the predicted supply/capacity without buffer, 
$d$ is the percentage value in decimal form, 
and $a$ is the absolute value of the buffer. 

Using a combination of this buffer capability with the 
installed capacity deployment driving method is 
effective in minimizing undersupply of a commodity without 
having excessive over supply. 
This is demonstrated in section . 

\subsection{Prediction Algorithms}
To effectively meet the objectives of \deploy, we looked to 
prediction analysis/time-forecasting techniques that are 
commonly used in other fields that use prediction analysis. 
This is a novel approach that has never been applied to 
\gls{NFCSim}s. 
\subsection{}