\section{Discussion}
% This should explore the significance of the results of the work, not repeat
% them. A combined Results and Discussion section is often appropriate. Avoid
% extensive citations and discussion of published literature.

By carefully selecting \deploy parameters, we are able to 
effectively automate setting up of transition scenarios for 
different evaluation groups. 
This is more efficient than the previous efforts that
required a user to manually calculate and use trial and error 
to set up the deployment scheme for the supporting fuel cycle 
facilities. 
Transition scenario simulations set up this way are sensitive 
to changes in the input parameters. 
This becomes an arduous process since a slight change in one 
of the input parameters would result in the need to recalculate 
the deployment scheme.  
% Ask roberto what troubles he had. 

By automating this process, the user can vary input parameters 
in the simulation and \deploy will automatically adjust the
deployment scheme to meet the new constraints. 