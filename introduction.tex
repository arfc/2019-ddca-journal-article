\section{Introduction}
% This is taken from my global paper (how to cite?/Should i paraphrase)
For many fuel cycle simulators, reactor facilities are automatically 
deployed to meet a user-defined power demand. 
However, it is up to the user to define a deployment scheme of 
supporting facilities to ensure that there is no gap in the supply 
chain that results in idle reactor capacity. 
Some users choose to set support facilities to have an infinite 
capacity to avoid this issue, but this is an inaccurate 
representation of reality. 
It is straightforward to manually determine a deployment scheme 
for a once-through fuel cycle, however, it is difficult to effectively 
implement for complex closed fuel cycle scenarios.  
To ease setting up of realistic \gls{NFC} simulations, a \gls{NFCSim}
should bring demand responsive deployment decisions into 
the dynamics of the simulation logic \cite{huff_current_2017}. 
Thus, a next generation \gls{NFCSim} should predictively and 
automatically deploy fuel cycle facilities to meet a user defined 
power demand. 

\Cyclus is an agent-based nuclear fuel cycle simulation framework 
\cite{huff_fundamental_2016}. 
In \Cyclus, each entity (i.e. Region, Institution, or Facility) in the 
fuel cycle is an agent. 
Region agents represent geographical or political areas that institution
and facility agents can be grouped into. 
Institution agents control the 
deployment and decommission of facility agents 
and represents legal operating organizations such as a 
utility, government, etc. \cite{huff_fundamental_2016}. 
Facility agents represent nuclear fuel cycle facilities. 
\Cycamore \cite{carlsen_cycamore_2014}
provides facility agents to represent process physics of various 
components in the nuclear fuel cycle (e.g. mine, fuel enrichment 
facility, reactor). 

\subsection{Context of Work}
An evaluation and screening study of a comprehensive set of nuclear 
\gls{FCO} \cite{wigeland_nuclear_2014} was conducted to assess 
for performance improvements compared the existing once-through 
fuel cycle (EG01) in the \gls{US} across a wide range of criteria. 
It was found that fuel cycles that consistently scored high 
overall performance involved continuous recycling
of co-extracted U/Pu or U/TRU in fast spectrum critical reactors. 
In the study, these fuel cycles were referred to as EG23, EG24, 
EG29 and EG30. 
Table \ref{tab:eg} provides a description of these fuel cycles. 

\floatsetup[table]{capposition=top}
\begin{table}[!htb]
    \begin{tabular}{|l|l|}
        \hline
        Fuel Cycle & Description                                                                                                                                 \\ \hline
        EG01 (current)      & \begin{tabular}[c]{@{}l@{}}Once-through using enriched-U fuel in \\ thermal critical reactors.\end{tabular}                                 \\ \hline
        EG23       & \begin{tabular}[c]{@{}l@{}}Continuous recycle of U/Pu with new\\ natural-U fuel in fast critical reactors.\end{tabular}                     \\ \hline
        EG24       & \begin{tabular}[c]{@{}l@{}}Continuous recycle of U/TRU with new\\ natural-U fuel in fast critical reactors.\end{tabular}                    \\ \hline
        EG29       & \begin{tabular}[c]{@{}l@{}}Continuous recycle of U/Pu with new\\ natural-U fuel in both fast and thermal\\ critical reactors.\end{tabular}  \\ \hline
        EG30       & \begin{tabular}[c]{@{}l@{}}Continuous recycle of U/TRU with new\\ natural-U fuel in both fast and thermal\\ critical reactors.\end{tabular} \\ \hline
    \end{tabular}
    \caption{Descriptions of the current and other high performing nuclear fuel cycle evaluation groups described in the evaluation and screening study \cite{wigeland_nuclear_2014}.}
    \label{tab:eg}
\end{table}

The evaluation and screening study assumed that 
the nuclear energy system was at an equilibrium to understand 
the end-state benefits of each evaluation group (EG). 
Based on the results from the study, the next step is 
to understand and evaluate the transition from the initial EG01
state to these promising future end-states 
\cite{feng_standardized_2016}. 
To successfully conduct analysis of the time-dependent transition
analyses, it is necessary to develop \gls{NFCSim} tools to  
automate setting up of transition scenarios. 
Therefore, Demand-Driven Cycamore Archetypes project
(NEUP-FY16-10512) was initiated to develop demand-driven deployment 
capabilities in \Cyclus. 

This capability is added as a \Cyclus Institution
agent that deploys facilities to meet the front-end and back-end 
fuel cycle demands based on a user-defined commodity demand. 
This demand-driven deployment capability is called 
\deploy. 

\subsection{Novelty}
To effectively predict supply and demand of commodities in 
\deploy, we looked to time series forecasting methods that are 
commonly used in other fields for making future predictions 
based on past time series data. 
This is a novel approach that has never been applied to 
\gls{NFCSim}s. 

\subsection{Objectives}
\label{sec:obj}
The main objectives of this paper are: 
(1) to describe the demand driven deployment capabilities of 
\Cyclus, (2) to describe the prediction methods available in 
\deploy, (3) to demonstrate the use of \deploy in setting up 
transition scenarios with various power demand curves, and (4) 
to compare setting up transition scenarios in \Cyclus with and 
without \deploy. 