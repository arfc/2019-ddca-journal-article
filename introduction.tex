\section{Introduction}
The nuclear fuel cycle represents the nuclear fuel life cycle from initial
extraction through processing, use in reactors, and, eventually, 
final disposal.
This complex system of facilities and mass flows 
collectively provide nuclear energy 
in the form of electricity \cite{yacout_modeling_2005}.
Nuclear fuel cycle simulator tools were introduced to investigate 
nuclear fuel cycle dynamics at a local and global level. 
These simulators track the flow of materials through the nuclear fuel cycle, 
from enrichment to final disposal of the fuel, while also accounting for 
decay and transmutation of isotopes. 
The impacts are evaluated in the form of `metrics', quantitative measures 
of performance \cite{huff_fundamental_2016}. 
These metrics are calculated from mass balances and facility operation 
histories calculated by a fuel cycle simulator \cite{huff_fundamental_2016}. 
By evaluating performance metrics of different fuel cycles, we gain an 
understanding of how each facility's parameters and technology choices 
impact the system's performance. 
Therefore, these results can be used to guide research 
efforts, advise future design choices, and provide 
decision-makers with a transparent tool for evaluating 
Fuel Cycle Options to inform policy decisions 
\cite{yacout_modeling_2005}.

Many fuel cycle simulators automatically deploy reactor facilities 
to meet a user-defined power demand. 
However, the user must define a deployment scheme of 
supporting facilities to avoid gaps in the supply 
chain resulting in idle reactor capacity. 
Current simulators require the user to set infinite capacity 
for supporting facilities but this inaccurately represents 
reality and obfuscates required capacities. 
Manually determining a deployment scheme for a once-through 
fuel cycle is straightforward, however, for complex fuel cycle 
scenarios, it is not.   
To ease setting up realistic nuclear fuel cycle simulations, a nuclear fuel cycle simulator
must bring dynamic demand-responsive deployment decisions into 
the simulation logic \cite{huff_current_2017}. 
This means the nuclear fuel cycle simulator decides how many mines, 
mills, enrichment facilities,
reprocessing facilities, etc are deployed to support dynamically 
changing power demand and reactor types.  
Thus, a next-generation nuclear fuel cycle simulator must predictively and 
automatically deploy fuel cycle facilities to meet a user-defined 
power demand. 

In \Cyclus, an agent-based nuclear fuel cycle simulation framework 
\cite{huff_fundamental_2016}, 
each entity (i.e. \texttt{Region}, \texttt{Institution}, or \texttt{Facility}) in the 
fuel cycle is an agent. 
\texttt{Region} agents represent geographical or political areas in which \texttt{Institution}
and \texttt{Facility} agents reside. 
\texttt{Institution} agents represent legal operating organizations such as
utilities, governments, and control the 
deployment and decommissioning of \texttt{Facility} agents
\cite{huff_fundamental_2016}.
\texttt{Facility} agents represent nuclear fuel cycle facilities
such as mines, conversion facilities, reactors, reprocessing facilities, 
etc. 
\Cycamore \cite{carlsen_cycamore_2014}
provides basic \texttt{Region}, \texttt{Institution}, 
and \texttt{Facility} archetypes compatible with \Cyclus. 

\subsection{Context of Work}
The impact of climate change on natural and human systems 
is increasingly apparent \cite{noauthor_climate_2018}.
The production and use of energy contribute to 
two-thirds of the total Green House Gas (GHG) 
emissions \cite{noauthor_climate_2018}.
Furthermore, as the human population increases and previously 
under-developed nations rapidly industrialize, 
global energy demand is forecasted to increase. 
Energy generation technology selection 
profoundly impacts climate change via growing energy demand. 
Large scale deployment of emissions free nuclear power plants 
could significantly reduce GHG production 
\cite{noauthor_climate_2018}.    

However, large scale nuclear power deployment faces
challenges of cost, safety, and used nuclear fuel  
\cite{petti_future_2018}. 
Nuclear power has high capital costs, 
an unresolved long-term nuclear waste management 
strategy and perceived adverse safety, environmental, and health 
effects \cite{petti_future_2018}. 
The nuclear power industry must overcome these challenges 
to ensure continued global use and expansion 
of nuclear energy technology. 

The challenges described above are associated with 
the present once-through fuel cycle in the \gls{US}, 
in which fabricated nuclear fuel is used once and placed into 
storage to await disposal. 
An evaluation and screening study of a comprehensive set of nuclear 
fuel cycle options \cite{wigeland_nuclear_2014} was conducted to assess 
for promising \glspl{EG} with performance improvements compared with 
the existing once-through 
fuel cycle (EG01) in the \gls{US} across a wide range of criteria. 
Fuel cycles that involved continuous recycling
of co-extracted U/Pu or U/TRU in fast spectrum critical reactors
consistently scored high on overall performance.  
Table \ref{tab:eg} describes these fuel cycles:
EG23, EG24, EG29, and EG30. 

    \begin{table}[]
        \centering
        \caption{Descriptions of the current and other high performing nuclear fuel cycle evaluation groups described in the evaluation and screening study \cite{wigeland_nuclear_2014}.}
        \label{tab:eg}
            \footnotesize
            \begin{tabular}{llll}
                \hline
            \textbf{Fuel Cycle}                                               & \textbf{Open or Closed} & \textbf{Fuel Type}                                                              & \textbf{Reactor Type}                                                                           \\ \hline
            \textbf{\begin{tabular}[c]{@{}l@{}}EG01\\ (current)\end{tabular}} & Open                                                               & Enriched-U                                                                      & Thermal                                                                       \\ 
            \textbf{EG23}                                                     & Closed                                                             & \begin{tabular}[c]{@{}l@{}}Recycled U/Pu \\ + Natural-U\end{tabular}  & Fast                                                                        \\ 
            \textbf{EG24}                                                     & Closed                                                             & \begin{tabular}[c]{@{}l@{}}Recycled U/TRU \\ + Natural-U\end{tabular} & Fast                                                                  \\ 
            \textbf{EG29}                                                     & Closed                                                             & \begin{tabular}[c]{@{}l@{}}Recycled U/Pu \\ + Natural-U\end{tabular}  & Fast \& Thermal  \\ 
            \textbf{EG30} & Closed                                                             & \begin{tabular}[c]{@{}l@{}}Recycled U/TRU \\ + Natural-U\end{tabular} & Fast \& Thermal  \\ \hline
        \end{tabular}
    \end{table}

The evaluation and screening study assumed
the nuclear energy systems were at equilibrium to understand 
the end-state benefits of each evaluation group \cite{feng_standardized_2016}. 
In the current work, our goal is to model the transition from the initial EG01
state to these promising future end-states without assuming equilibrium
fuel cycles. 
To successfully analyze time-dependent transition
scenarios, the nuclear fuel cycle simulator tool must 
automate the transition scenario simulation setup. 
Therefore, the Demand-Driven \Cycamore Archetypes project
(NEUP-FY16-10512) was initiated to develop 
demand-driven deployment capabilities in \Cyclus. 
This capability, \deploy, is a \Cyclus \texttt{Institution}
agent that deploys facilities to meet user-defined power demand. 

\subsection{Novelty}
We utilized time series forecasting methods to effectively predict 
future commodities' supply and demand in \deploy. 
Solar and wind power generation commonly use these methods
to make future predictions based on past time series data
\cite{reikard_predicting_2009,diagne_review_2013,soman_review_2010,taylor_wind_2009}. 
Industrial supply chain management also uses sophisticated time series 
forecasting techniques to predict demand for quantities of goods 
in the supply chain \cite{souza_supply_2014}.
This is a novel approach that has never been applied to 
nuclear fuel cycle simulators. 

\subsection{Objectives}
\label{sec:obj}
The main objectives of this paper are: 
(1) to describe the demand-driven deployment capabilities in 
\Cyclus, 
(2) to describe the prediction methods available in 
\deploy, and
(3) to demonstrate the use of \deploy in setting up 
EG01-23, EG01-24, EG01-29, and EG01-30 transition scenarios 
with various power demand curves.