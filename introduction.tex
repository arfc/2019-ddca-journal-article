\section{Introduction}
\gls{NFC} simulators are system analysis tools used to evaluate 
quantitative measures of dynamic \gls{NFC} performance 
in both high and low resolution. 
Plutonium concentration in a single used fuel bundle and 
total electricity produced are examples of high and low 
resolution elements respectively.  
The primary purpose of \gls{NFC} simulators is to understand the 
dependence between various input parameters and components 
in the \gls{NFC} and the impact their variations have on 
the system's performance. 
The results of \gls{NFC} simulators are used to guide research 
efforts, advise future design choices, and provide 
decision-makers with a transparent tool for evaluating \gls{FCO} 
to inform big-picture policy decisions \cite{yacout_modeling_2005}.

Many fuel cycle simulators, automatically deploy reactor facilities 
to meet a user-defined power demand. 
However, the user must define a deployment scheme of 
supporting facilities to avoid gaps in the supply 
chain resulting in idle reactor capacity. 
To avoid this issue, some users choose to set supporting 
facilities to have an infinite capacity, but this is an inaccurate 
representation of reality resulting in misrepresented results. 
It is straightforward to manually determine a deployment scheme 
for a once-through fuel cycle, however, it is not straightforward
for complex closed fuel cycle scenarios.  
To ease setting up realistic \gls{NFC} simulations, a \gls{NFC} simulator
should bring demand-responsive deployment decisions into 
the simulation logic dynamics \cite{huff_current_2017}. 
Thus, a next-generation \gls{NFC} simulator should predictively and 
automatically deploy fuel cycle facilities to meet a user defined 
power demand. 

In \Cyclus, an agent-based nuclear fuel cycle simulation framework 
\cite{huff_fundamental_2016}, 
each entity (i.e. \texttt{Region}, \texttt{Institution}, or \texttt{Facility}) in the 
fuel cycle is an agent. 
\texttt{Region} agents represent geographical or political areas that \texttt{Institution}
and \texttt{Facility} agents reside. 
\texttt{Institution} agents control the 
deployment and decommission of \texttt{Facility} agents 
and represent legal operating organizations such as a 
utilities, governments, etc. \cite{huff_fundamental_2016}. 
\texttt{Facility} agents represent nuclear fuel cycle facilities
such as mines, conversion facilities, reactors, reprocessing facilities, 
etc. 
\Cycamore \cite{carlsen_cycamore_2014}
provides basic \Cyclus' \texttt{Region}, \texttt{Institution}, 
and \texttt{Facility} archetypes. 

\subsection{Context of Work}
The impact of climate change on natural and human systems 
is increasingly apparent.
The production and use of energy contributes to 
two-thirds of the total Green House Gas (GHG) 
emissions \cite{noauthor_climate_2018}.
Furthermore, as human population increases and previously 
under-developed nations urbanize rapidly, 
global energy demand is forecasted to increase.  
The types of power generation technologies used 
will heavily impact the effects of growing energy demand 
on climate change.  
Large scale deployment of nuclear power plants has significant 
potential to reduce GHG production due to their low 
carbon emissions \cite{noauthor_climate_2018}.  

However, the nuclear power industry is facing four major challenges 
of large scale nuclear power deployment: 
cost, safety, proliferation, and waste 
\cite{massachusetts_institute_of_technology_future_2003}. 
Nuclear power has high overall lifetime costs and increases 
risks of nuclear proliferation. 
There is also an unresolved long-term nuclear waste management 
strategy and perceived adverse safety, environmental, and health 
effects \cite{massachusetts_institute_of_technology_future_2003}. 
The nuclear power industry must overcome these four challenges 
to ensure continued global use and expansion 
of nuclear energy technology. 

The four challenges described above are associated with 
the present once-through fuel cycle in the \gls{US}, 
in which fabricated nuclear fuel is used once and placed into 
storage to await disposal. 
An evaluation and screening study of a comprehensive set of nuclear 
fuel cycle options \cite{wigeland_nuclear_2014} was conducted to assess 
for performance improvements compared with the existing once-through 
fuel cycle (EG01) in the \gls{US} across a wide range of criteria. 
Fuel cycles that involved continuous recycling
of co-extracted U/Pu or U/TRU in fast spectrum critical reactors
consistently scored high on overall performance.  
Table \ref{tab:eg} describes these fuel cycles:
EG23, EG24, EG29, and EG30. 

    \begin{table}[]
        \centering
        \caption{Descriptions of the current and other high performing nuclear fuel cycle evaluation groups described in the evaluation and screening study \cite{wigeland_nuclear_2014}.}
        \label{tab:eg}
            \footnotesize
            \begin{tabularx}{\textwidth}{l|lll}
                \hline
            \textbf{Fuel Cycle}                                               & \textbf{Open or Closed} & \textbf{Fuel Type}                                                              & \textbf{Reactor Type}                                                                           \\ \hline
            \textbf{\begin{tabular}[c]{@{}l@{}}EG01\\ (current)\end{tabular}} & Open                                                               & Enriched-U                                                                      & Thermal Critical                                                                       \\ 
            \textbf{EG23}                                                     & Closed                                                             & \begin{tabular}[c]{@{}l@{}}Recycled U/Pu \\ + Natural-U\end{tabular}  & Fast Critical                                                                         \\ 
            \textbf{EG24}                                                     & Closed                                                             & \begin{tabular}[c]{@{}l@{}}Recycled U/TRU \\ + Natural-U\end{tabular} & Fast Critical                                                                   \\ 
            \textbf{EG29}                                                     & Closed                                                             & \begin{tabular}[c]{@{}l@{}}Recycled U/Pu \\ + Natural-U\end{tabular}  & Fast Critical \& Thermal Critical  \\ 
            \textbf{EG30} & Closed                                                             & \begin{tabular}[c]{@{}l@{}}Recycled U/TRU \\ + Natural-U\end{tabular} & Fast Critical \& Thermal Critical  \\ \hline
        \end{tabularx}
    \end{table}

The evaluation and screening study assumed
the nuclear energy systems were at equilibrium to understand 
the end-state benefits of each \gls{EG} \cite{feng_standardized_2016}. 
In this work, our goal is to model the transition from the initial EG01
state to these promising future end-states. 
To successfully analyze time-dependent transition
scenarios, the \gls{NFC} simulator tool must 
automate transition scenario simulation setup. 
Therefore, the Demand-Driven \Cycamore Archetypes project
(NEUP-FY16-10512) was initiated to develop 
demand-driven deployment capabilities in \Cyclus. 
This capability, \deploy, is a \Cyclus \texttt{Institution}
agent that deploys facilities to meet a user-defined power demand. 

\subsection{Novelty}
We utilized time series forecasting methods to effectively predict 
future commodity supply and demand in \deploy. 
Solar and wind power generation commonly use these methods
to make future predictions based on past time series data
\cite{reikard_predicting_2009,diagne_review_2013,soman_review_2010,taylor_wind_2009}. 
This is a novel approach that has never been applied to 
\gls{NFC} simulators. 

\subsection{Objectives}
\label{sec:obj}
The main objectives of this paper are: 
(1) to describe the demand-driven deployment capabilities in 
\Cyclus, 
(2) to describe the prediction methods available in 
\deploy, and
(3) to demonstrate the use of \deploy in setting up 
EG01-23, EG01-24, EG01-29, and EG01-30 transition scenarios 
with various power demand curves.