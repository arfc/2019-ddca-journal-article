%        File: revise.tex
%     Created: Wed Oct 09 02:00 PM 2013 P
% Last Change: Wed Oct 09 02:00 PM 2013 P
%

%
% Copyright 2007, 2008, 2009 Elsevier Ltd
%
% This file is part of the 'Elsarticle Bundle'.
% ---------------------------------------------
%
% It may be distributed under the conditions of the LaTeX Project Public
% License, either version 1.2 of this license or (at your option) any
% later version.  The latest version of this license is in
%    http://www.latex-project.org/lppl.txt
% and version 1.2 or later is part of all distributions of LaTeX
% version 1999/12/01 or later.
%
% The list of all files belonging to the 'Elsarticle Bundle' is
% given in the file `manifest.txt'.
%

% Template article for Elsevier's document class `elsarticle'
% with numbered style bibliographic references
% SP 2008/03/01
%
%
%
% $Id: elsarticle-template-num.tex 4 2009-10-24 08:22:58Z rishi $
%
%
%\documentclass[preprint,12pt]{elsarticle}
\documentclass[answers,11pt]{exam}

% \documentclass[preprint,review,12pt]{elsarticle}

% Use the options 1p,twocolumn; 3p; 3p,twocolumn; 5p; or 5p,twocolumn
% for a journal layout:
% \documentclass[final,1p,times]{elsarticle}
% \documentclass[final,1p,times,twocolumn]{elsarticle}
% \documentclass[final,3p,times]{elsarticle}
% \documentclass[final,3p,times,twocolumn]{elsarticle}
% \documentclass[final,5p,times]{elsarticle}
% \documentclass[final,5p,times,twocolumn]{elsarticle}

% if you use PostScript figures in your article
% use the graphics package for simple commands
% \usepackage{graphics}
% or use the graphicx package for more complicated commands
\usepackage{graphicx}
% or use the epsfig package if you prefer to use the old commands
% \usepackage{epsfig}

% The amssymb package provides various useful mathematical symbols
\usepackage{amssymb}
% The amsthm package provides extended theorem environments
% \usepackage{amsthm}
\usepackage{amsmath}

% The lineno packages adds line numbers. Start line numbering with
% \begin{linenumbers}, end it with \end{linenumbers}. Or switch it on
% for the whole article with \linenumbers after \end{frontmatter}.
\usepackage{lineno}

% I like to be in control
\usepackage{placeins}

% natbib.sty is loaded by default. However, natbib options can be
% provided with \biboptions{...} command. Following options are
% valid:

%   round  -  round parentheses are used (default)
%   square -  square brackets are used   [option]
%   curly  -  curly braces are used      {option}
%   angle  -  angle brackets are used    <option>
%   semicolon  -  multiple citations separated by semi-colon
%   colon  - same as semicolon, an earlier confusion
%   comma  -  separated by comma
%   numbers-  selects numerical citations
%   super  -  numerical citations as superscripts
%   sort   -  sorts multiple citations according to order in ref. list
%   sort&compress   -  like sort, but also compresses numerical citations
%   compress - compresses without sorting
%
% \biboptions{comma,round}

% \biboptions{}


% Katy Huff addtions
\usepackage{xspace}
\usepackage{color}
\usepackage[hyphens]{url}

\newcommand{\Cyclus}{\textsc{Cyclus}\xspace}%
\newcommand{\Cycamore}{\textsc{Cycamore}\xspace}%
\newcommand{\deploy}{\texttt{d3ploy}\xspace}%

%\journal{Annals of Nuclear Energy}

\begin{document}

%\begin{frontmatter}

   % Title, authors and addresses

   % use the tnoteref command within \title for footnotes;
   % use the tnotetext command for the associated footnote;
   % use the fnref command within \author or \address for footnotes;
   % use the fntext command for the associated footnote;
   % use the corref command within \author for corresponding author footnotes;
   % use the cortext command for the associated footnote;
   % use the ead command for the email address,
   % and the form \ead[url] for the home page:
   %
   % \title{Title\tnoteref{label1}}
   % \tnotetext[label1]{}
   % \author{Name\corref{cor1}\fnref{label2}}
   % \ead{email address}
   % \ead[url]{home page}
   % \fntext[label2]{}
   % \cortext[cor1]{}
   % \address{Address\fnref{label3}}
   % \fntext[label3]{}

\title{Demand Driven Deployment Capabilities in Cyclus, a Fuel Cycle Simulator}

   % use optional labels to link authors explicitly to addresses:
   % \author[label1,label2]{<author name>}
   % \address[label1]{<address>}
   % \address[label2]{<address>}


%\author[uiuc]{Kathryn Huff}
%        \ead{kdhuff@illinois.edu}
%  \address[uiuc]{Department of Nuclear, Plasma, and Radiological Engineering,
%        118 Talbot Laboratory, MC 234, Universicy of Illinois at
%        Urbana-Champaign, Urbana, IL 61801}
%
% \end{frontmatter}

\section*{Review General Response}
We would like to again thank the reviewers for their detailed assessment of
this paper.


\section*{Reviewer 1}
\begin{questions}

\question General. This paper applies the well-known fuel cycle simulator Cyclus to analyze the DOE nuclear
fuel cycle options. Cyclus is widely recognized as a flexible, fuel cycle analysis computational tool designed for problems such as this. The literature review is satisfactory for the scope of the work. Thank
you for including line numbers. It’s surprising how many manuscripts do not include them especially
since inclusion is not particularly labor intensive. This is a minor criticism, but this reads like a thesis/report. A journal paper is a little more streamlined and focused since the authors do not have to
prepare a defense for their work.

\begin{solution}
        Thank you for your kind review.
        I have streamlined the paper to make it more focused to read less like a thesis/report. 
        See the responses below for more detail..
\end{solution}

\question Section 3. This section turned into largely an information dump. Again, possibly due to the report v
journal paper issue. There is a lack of the ‘so what’ that is needed for journal papers. It’s not easy,
especially with so many figures. There is a lot of ‘performs best’ being thrown around, but not really
what that means. I don’t know if it would be better to segregate the results into further subsections, but
the impact of the work risks being lost. Don’t assume the reader is going to understand the implications
just because

\begin{solution}
\end{solution}

\question Figures. This might be a bit picky, but is there a way to line the figures up more with the text? For
instance, Figures 7,8,9 are referenced on p21, but you have to scroll down several pages to see them. As
someone who includes lots of graphs in their research too, this is understandably a challenge.

\begin{solution}
\end{solution}

\question Conclusion. This section is rather glib, given what seems like a quite a body of work that was produced.
I would recommend to include major findings and implications clearly.

\begin{solution}
\end{solution}

\question Future work. Similarly, after all this work, only a sensitivity analysis is suggested. Is there anything
more? What is envisioned the long term use of d3ploy?

\begin{solution}
\end{solution}

\question Acknowledgments. I don’t know if it’s necessary to include author contributions. Given that Prof. Huff
is the author of record; i.e., listed last, it is known she directed the work, and given her reputation, there
is no doubt anyone listed as an author contributed meaningfully to the work. I’ve never really seen that
in journal papers anyway, but authors’ discretion.

\begin{solution}
\end{solution}

\question Line Items (Abstract)

1) Abstract - Not everyone is going to know what d3ploy is. Either elaborate (a short sentence) or
remove it and just explain it later on in the paper. \\
2) Abstract - The claim of ‘more efficient’ should be supported by a clearer context; i.e., efficient in
what way?


\begin{solution}
The abstract has been modified to read: \\
The demand-driven deployment capabilities are refered to as \deploy. 
We demonstrate \deploy's capability to predict future commodities' 
supply and demand, and automatically deploy fuel 
cycle facilities to meet the predicted demand in four transition scenarios. 
Using \deploy to set up transition scenarios saves the user 
simulation set-up time compared to previous efforts that
required a user to manually calculate and use trial and error 
to set up the deployment scheme for the supporting fuel cycle 
facilities. 
\end{solution}

\question Line Items (Non-quick fixes)
3) 3-30 - While there are certainly many people familiar with the capabilities of Cyclus, there may be
some who are not. It might be instructive for some more description of it, either here or in a separate
section. Only a paragraph or two, maximum. Some newer readers might not know what agent-based
means.

\question Line Items (Non-quick fixes)
5) 3-58 - Ref. 7 is now 6 years old. Are these fuel cycle options still considered DOE policy? In and of
themselves, these are acceptable options for study with Cyclus, but it still begs the comment as to its
status.

\question Line Items (Non-quick fixes)
6) 4-63 - What does performance mean in this context?

\question Line Items (Non-quick fixes)

\begin{solution}
\end{solution}

\end{questions}

\section*{Reviewer 2}
\begin{questions}

\question Comments on Content.
You make no mention of the impact of the issue of dynamic fuel compositions. 
The commodity that the reactors are demanding and the one being supplied 
by the reprocessing plants is constantly in flux during a transition 
scenario with unlimited recycle. Other applications that use forecasting 
methods don't have this concern.  So effectively what you need to predict 
isn't just the capacities needed based on the mass of SNF, but also its 
post-reprocessing worth. When using reprocessed fuel, the necessary fissile 
loading fraction of MOX or TRU in a fast reactor may be as much as 50\% 
higher for material sourced from a MOX LWR than from a UOX LWR. 
This difference will significantly impact your reprocessing capacity 
required to supply that material.

\begin{solution}
        Thank you for your kind review.
        The statement concerning `unmatched' fidelity has been removed
        entirely and the bibliography has been expanded significantly. See the
        responses below for more detail..
\end{solution}

\question 

Pg 2. Fuel cycle options doesn't need to be capitalized. \\
Pg 3. Greenhouse is one word and greenhouse gas doesn't need to be capitalized. \\
Pg 3. Line 50-52. This sentence is hard to follow with nested lists and clunky grammar.\\
Pg 3. Line 59 Evaluation groups doesn't need to be capitalized. \\
Pg 9. This section might be easier to follow if it was an enumerated list rather than forcing a paragraph structure.\\
Pg 9. You use the acronym LWR but don't define it until page 11.\\
Pg 10. Please use the same indentation for both equations 4 and 5.\\
Pg 12. Numbers on axes for figure 3 are small.\\
Pg 13. The definition of terms in an equation shouldn't have its own equation number. Equation 7 should be a list or part of equation 6. Also, please match indentation.\\
Pg 15. Please fix indentation after equation 9.\\
Pg 15. The sentence after an equation does not always start a new paragraph, so it shouldn't be indented if it doesn't.\\
Pg 16. Equation 12 should be a list or part of equation 11. \\
Pg 16. You do not define what the "L" term is in equation 13.\\
Pg 17. You do not define what the "d" or "Y" terms are in equation 14.\\
Pg 18. Rather than starting the results section with a sentence 7 lines long, it would improve readability to make it into an enumerated list.\\
Pg 19. You don't need to state in the caption to figure 5 that power undersupply is avoided. You state that in the preceding paragraph
Pg 23. The labels and axes are too small in figure 8. You also don't need to state results in your caption that are already stated in the main text.\\
Pg 24. The labels and axes are too small in figure 9. You also don't need to state results in your caption that are already stated in the main text.\\
\begin{solution}
All these comments have been addressed and fixed. 
\end{solution}

\question
Pg 5. It is unclear if d3ploy runs before the simulation or if it is 
doing these calculation on-the-fly. If you are only predicting the 
necessary capacity a single time-step in advance does that mean that 
there is no deployment or process times?

\end{questions}
\bibliographystyle{unsrt}
\bibliography{../2019-d3ploy-transition.bib}
  \end{document}

  %
  % End of file `elsarticle-template-num.tex'.
