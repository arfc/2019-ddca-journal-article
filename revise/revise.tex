%        File: revise.tex
%     Created: Wed Oct 09 02:00 PM 2013 P
% Last Change: Wed Oct 09 02:00 PM 2013 P
%

%
% Copyright 2007, 2008, 2009 Elsevier Ltd
%
% This file is part of the 'Elsarticle Bundle'.
% ---------------------------------------------
%
% It may be distributed under the conditions of the LaTeX Project Public
% License, either version 1.2 of this license or (at your option) any
% later version.  The latest version of this license is in
%    http://www.latex-project.org/lppl.txt
% and version 1.2 or later is part of all distributions of LaTeX
% version 1999/12/01 or later.
%
% The list of all files belonging to the 'Elsarticle Bundle' is
% given in the file `manifest.txt'.
%

% Template article for Elsevier's document class `elsarticle'
% with numbered style bibliographic references
% SP 2008/03/01
%
%
%
% $Id: elsarticle-template-num.tex 4 2009-10-24 08:22:58Z rishi $
%
%
%\documentclass[preprint,12pt]{elsarticle}
\documentclass[answers,11pt]{exam}

% \documentclass[preprint,review,12pt]{elsarticle}

% Use the options 1p,twocolumn; 3p; 3p,twocolumn; 5p; or 5p,twocolumn
% for a journal layout:
% \documentclass[final,1p,times]{elsarticle}
% \documentclass[final,1p,times,twocolumn]{elsarticle}
% \documentclass[final,3p,times]{elsarticle}
% \documentclass[final,3p,times,twocolumn]{elsarticle}
% \documentclass[final,5p,times]{elsarticle}
% \documentclass[final,5p,times,twocolumn]{elsarticle}

% if you use PostScript figures in your article
% use the graphics package for simple commands
% \usepackage{graphics}
% or use the graphicx package for more complicated commands
\usepackage{graphicx}
% or use the epsfig package if you prefer to use the old commands
% \usepackage{epsfig}

% The amssymb package provides various useful mathematical symbols
\usepackage{amssymb}
% The amsthm package provides extended theorem environments
% \usepackage{amsthm}
\usepackage{amsmath}

% The lineno packages adds line numbers. Start line numbering with
% \begin{linenumbers}, end it with \end{linenumbers}. Or switch it on
% for the whole article with \linenumbers after \end{frontmatter}.
\usepackage{lineno}

% I like to be in control
\usepackage{placeins}

% natbib.sty is loaded by default. However, natbib options can be
% provided with \biboptions{...} command. Following options are
% valid:

%   round  -  round parentheses are used (default)
%   square -  square brackets are used   [option]
%   curly  -  curly braces are used      {option}
%   angle  -  angle brackets are used    <option>
%   semicolon  -  multiple citations separated by semi-colon
%   colon  - same as semicolon, an earlier confusion
%   comma  -  separated by comma
%   numbers-  selects numerical citations
%   super  -  numerical citations as superscripts
%   sort   -  sorts multiple citations according to order in ref. list
%   sort&compress   -  like sort, but also compresses numerical citations
%   compress - compresses without sorting
%
% \biboptions{comma,round}

% \biboptions{}


% Katy Huff addtions
\usepackage{xspace}
\usepackage{color}
\usepackage[hyphens]{url}

\newcommand{\Cyclus}{\textsc{Cyclus}\xspace}%
\newcommand{\Cycamore}{\textsc{Cycamore}\xspace}%
\newcommand{\deploy}{\texttt{d3ploy}\xspace}%

%\journal{Annals of Nuclear Energy}

\begin{document}

%\begin{frontmatter}

   % Title, authors and addresses

   % use the tnoteref command within \title for footnotes;
   % use the tnotetext command for the associated footnote;
   % use the fnref command within \author or \address for footnotes;
   % use the fntext command for the associated footnote;
   % use the corref command within \author for corresponding author footnotes;
   % use the cortext command for the associated footnote;
   % use the ead command for the email address,
   % and the form \ead[url] for the home page:
   %
   % \title{Title\tnoteref{label1}}
   % \tnotetext[label1]{}
   % \author{Name\corref{cor1}\fnref{label2}}
   % \ead{email address}
   % \ead[url]{home page}
   % \fntext[label2]{}
   % \cortext[cor1]{}
   % \address{Address\fnref{label3}}
   % \fntext[label3]{}

\title{Demand Driven Deployment Capabilities in Cyclus, a Fuel Cycle Simulator}

   % use optional labels to link authors explicitly to addresses:
   % \author[label1,label2]{<author name>}
   % \address[label1]{<address>}
   % \address[label2]{<address>}


%\author[uiuc]{Kathryn Huff}
%        \ead{kdhuff@illinois.edu}
%  \address[uiuc]{Department of Nuclear, Plasma, and Radiological Engineering,
%        118 Talbot Laboratory, MC 234, Universicy of Illinois at
%        Urbana-Champaign, Urbana, IL 61801}
%
% \end{frontmatter}

\section*{Review General Response}
We would like to again thank the reviewers for their detailed assessment of
this paper. Your comments have resulted in changes which certainly improved the 
paper.


\section*{Reviewer 1}
\begin{questions}

\question General. This paper applies the well-known fuel cycle simulator Cyclus to analyze the DOE nuclear
fuel cycle options. Cyclus is widely recognized as a flexible, fuel cycle analysis computational tool designed for problems such as this. The literature review is satisfactory for the scope of the work. Thank
you for including line numbers. It’s surprising how many manuscripts do not include them especially
since inclusion is not particularly labor intensive. This is a minor criticism, but this reads like a thesis/report. A journal paper is a little more streamlined and focused since the authors do not have to
prepare a defense for their work.

\begin{solution}
        Thank you for your kind review.
        We have restructured section III of the paper to make it more streamlined and focused. 
        See the responses below for more detail..
\end{solution}

\question Section 3. This section turned into largely an information dump. Again, possibly due to the report v
journal paper issue. There is a lack of the ‘so what’ that is needed for journal papers. It’s not easy,
especially with so many figures. There is a lot of ‘performs best’ being thrown around, but not really
what that means. I don’t know if it would be better to segregate the results into further subsections, but
the impact of the work risks being lost. Don’t assume the reader is going to understand the implications
just because

\begin{solution}
We have rearranged Section III to more clearly define how the results are broken down in each subsection. 
We clarified and enhanced the beginning description of the results section. The description has been 
modified to read: 

`This section aims to demonstrate \deploy's capability to completely automate the setup of 
transition scenarios and meet the objectives described in section 
I.C. 
This section is split into two subsections. 
The first subsection (section III.A) will demonstrate \deploy's capabilities 
to automate set up a simple transition scenario with only three facility types. 
The second subsection (section III/B) will demonstrate \deploy's capabilities to automate 
set up of complex EG01-23, EG01-24, EG01-29, EG01-30 transition scenarios and is further 
subdivided into:  
\begin{enumerate}
\item Section III.B.1: compare the use of different \deploy prediction methods in EG01-EG23, EG01-EG24, 
EG01-EG29, and EG01-EG30 transition scenarios, 
\item Section III.B.2: compare the use of varied power buffer sizes in EG01-EG23, EG01-EG24, 
EG01-EG29, and EG01-EG30 transition scenarios, and
\item Section III.B.3: demonstrate successful \deploy setup of EG01-EG23, EG01-EG24, 
EG01-EG29, and EG01-EG30 transition scenarios using the prediction method and 
power buffer size that proved to best minimize power undersupply in the Sections 
III.B.1 and III.B.2. 
These will be referred to as `best performance models'. 
\end{enumerate}
The input files and scripts to reproduce the results and plots in this
paper are found in \cite{chee_arfc/d3ploy:_2019} and 
\cite{bae_arfctransition-scenarios_2019}. '
\\

We have also removed instances of `performs best' and replaced them with specifics. 
\end{solution}

\question Figures. This might be a bit picky, but is there a way to line the figures up more with the text? For
instance, Figures 7,8,9 are referenced on p21, but you have to scroll down several pages to see them. As
someone who includes lots of graphs in their research too, this is understandably a challenge.

\begin{solution}
As a result of the rearrangement of the results section, Figure 7 has been moved to pg 23, after 
it is mentioned on pg 22. Figure 8 and 9 are mentioned on pg 24, and are shown on pg 25 and 26.  
\end{solution}

\question Conclusion. This section is rather glib, given what seems like a quite a body of work that was produced.
I would recommend to include major findings and implications clearly.

\begin{solution}
We have extended the conclusion to more clearly describe the major findings and implications of 
\deploy: 

The present nuclear fuel cycle in the United States is a once-through 
fuel cycle of LWRs with no used fuel reprocessing. 
This nuclear fuel cycle faces cost, safety, proliferation, and spent 
nuclear fuel challenges that hinder large-scale nuclear power deployment. 
The U.S Department of Energy identified future 
nuclear fuel cycles, involving continuous recycling of co-extracted U/Pu or 
U/TRU in fast and thermal spectrum reactors, that may overcome these challenges.
These transition scenarios have been modeled previously in the following 
nuclear fuel cycle simulators \cite{feng_standardized_2016,bae_standardized_2019}: 
ORION, DYMOND, VISION, MARKAL, and \Cyclus. 
However, for many nuclear fuel cycle simulators, the user is required to 
define a deployment scheme for all supporting facilities to avoid any 
supply chain gaps or resulting idle reactor capacity. 
Manually determining a deployment scheme for a once-through 
fuel cycle is straightforward; however, for complex fuel cycle 
scenarios, it is not. 
In this paper, we introduce the capability, \deploy, in \Cyclus 
that automatically deploys 
fuel cycle facilities to meet user-defined power demand.     
In this paper, we demonstrate that with careful selection of 
\deploy parameters, we can 
completely automate the setup of constant and 
linearly increasing power demand transition scenarios
for EG01-23, EG01-24, EG01-29, and EG01-30 with minimal 
power undersupply. 
Using \deploy to set up transition scenarios 
saves the user simulation set-up time, making it more efficient 
than the previous efforts that
required a user to manually calculate and use trial and error 
to set up the deployment scheme for the supporting fuel cycle 
facilities. 
Transition scenario simulations set up manually are sensitive 
to changes in the input parameters resulting in an 
arduous setup process since a slight change in one 
input parameter would result in the need to recalculate 
the deployment scheme to ensure no undersupply 
of power.   
Therefore, by automating this process, when the user varies input parameters 
in the simulation, \deploy automatically adjusts the
deployment scheme to meet the new constraints. 
\end{solution}

\question Future work. Similarly, after all this work, only a sensitivity analysis is suggested. Is there anything
more? What is envisioned the long term use of d3ploy?

\begin{solution}
The future works passage has been modified to read: 

We simulate transition scenarios to predict the future; 
however, when implemented in the real world, the transition 
scenario tends to deviate from the optimal scenario.
Therefore, nuclear fuel cycle simulators must be used to conduct
sensitivity analysis studies to understand the subtleties of 
a transition scenario better to reliably inform policy decisions.
Previously it was difficult to conduct sensitivity analysis with \Cyclus 
as users have to manually calculate the deployment scheme for a 
single change in an input parameter. 
By using the \deploy capability,
sensitivity analysis studies are more efficiently 
conducted as facility deployment in transition scenarios 
are automatically set up. 
\deploy will also be open-source and available for the forseeable future on github 
\cite{chee_arfc/d3ploy:_2019}, to be used with \Cyclus for conducting any 
transition scenario analysis. 
The transition scenario simulations in this work assume recipe reactors, 
however, in reality, complexity introduced by the reprocessing plants causes 
each reactor to have dynamic incoming and outgoing material compositions. 
Therefore, the static recipe method assumption is a poor approximation 
\cite{bae_neural_2019,peterson-droogh_value_2018}. 
In future transition scenario work with \deploy, a \Cyclus 
reactor archetype that uses dynamic fuel compositions could be used. 
\end{solution}

\question Acknowledgments. I don’t know if it’s necessary to include author contributions. Given that Prof. Huff
is the author of record; i.e., listed last, it is known she directed the work, and given her reputation, there
is no doubt anyone listed as an author contributed meaningfully to the work. I’ve never really seen that
in journal papers anyway, but authors’ discretion.

\begin{solution}
        We will leave it as is. 
\end{solution}

\question Line Items (Abstract)

1) Abstract - Not everyone is going to know what d3ploy is. Either elaborate (a short sentence) or
remove it and just explain it later on in the paper. \\
2) Abstract - The claim of ‘more efficient’ should be supported by a clearer context; i.e., efficient in
what way?


\begin{solution}
The abstract has been modified to read: 

The present United States' nuclear fuel cycle faces challenges that hinder 
the expansion of nuclear energy technology. 
The U.S. Department of Energy identified four nuclear fuel cycle 
options, which make nuclear energy technology
more desirable. 
Successfully analyzing the transitions from the current 
fuel cycle to these promising fuel cycles requires a nuclear 
fuel cycle simulator that can predictively and automatically 
deploy fuel cycle facilities to meet user-defined power demand. 
This work introduces and demonstrates demand-driven deployment 
capabilities in \Cyclus, an open-source nuclear fuel cycle simulator framework.  
User-controlled capabilities such as time series forecasting algorithms, supply buffers, 
and facility preferences were introduced to give users tools to minimize power 
undersupply in a transition scenario simulation. 
The demand-driven deployment capabilities are referred to as \deploy. 
We demonstrate \deploy's capability to predict future commodities' 
supply and demand, and automatically deploy fuel 
cycle facilities to meet the predicted demand in four transition scenarios. 
Using \deploy to set up transition scenarios saves the user 
simulation set-up time compared to previous efforts that
required a user to manually calculate and use trial and error 
to set up the deployment scheme for the supporting fuel cycle 
facilities. 
\end{solution}

\question Line Items \\
3) 3-30 - While there are certainly many people familiar with the capabilities of Cyclus, there may be
some who are not. It might be instructive for some more description of it, either here or in a separate
section. Only a paragraph or two, maximum. Some newer readers might not know what agent-based
means.

\begin{solution}
We have expanded the \Cyclus description. The passage now reads: 

\Cyclus is an agent-based nuclear fuel cycle simulation framework 
\cite{huff_fundamental_2016}, 
each entity (i.e. \texttt{Region}, \texttt{Institution}, or \texttt{Facility}) in the 
fuel cycle is an agent. 
An agent-based model enables model development to take place at an agent level 
rather than a system level \cite{huff_fundamental_2016}. 
For example, an analyst can design a reactor agent that is entirely independent 
from an fuel fabrication agent. Each agent's behavior is designed according to the 
application interface contract, giving them the capability to interact with each 
other in the simulation \cite{huff_fundamental_2016}.  
\texttt{Region} agents represent geographical or political areas in which \texttt{Institution}
and \texttt{Facility} agents reside. 
\texttt{Institution} agents represent legal operating organizations such as
utilities, governments, and control the 
deployment and decommissioning of \texttt{Facility} agents
\cite{huff_fundamental_2016}.
\texttt{Facility} agents represent nuclear fuel cycle facilities
such as mines, conversion facilities, reactors, reprocessing facilities, 
etc. 
\Cycamore \cite{carlsen_cycamore_2014}
provides basic \texttt{Region}, \texttt{Institution}, 
and \texttt{Facility} archetypes compatible with \Cyclus. 
A complete introduction to \Cyclus can be found in \cite{huff_fundamental_2016}. 
\end{solution}

\question Line Items \\
4) 3-51 - It’s not clear why the perceived adverse safety, etc., needs to be included. Is Cyclus going
to address these issues? (rhetorical) The point being if the paper isn’t going to show the research,
there really isn’t a need to include it since this is a nuclear engineering journal. The nuclear power
industry may not necessarily have to overcome the perceived problems if there was a coherent energy
policy established in the USA, and while that is a good discussion to have, it’s probably not part of
this paper.

\begin{solution}
Excellent point. We have removed the sentence about perceived adverse safety. You're right, the real purpose 
of these tools is not interaction with the public so much as providing a tool to potentially drive the 
R\&D directions that are taken by our leadership (DOE) as stated in 3-35. 
\end{solution}

\question Line Items \\
5) 3-58 - Ref. 7 is now 6 years old. Are these fuel cycle options still considered DOE policy? In and of
themselves, these are acceptable options for study with Cyclus, but it still begs the comment as to its
status.

\begin{solution}
More recent citations can be found to bolster this. We have added the following text: 
`Recent statements from Rita Baranwal \cite{noauthor_new_2019}, the Nuclear Energy Innovation 
Capabilities Act \cite{crapo_s97_2018}, and the Advanced Nuclear Technology Development Act \cite{latta_hr590_2017} 
show that there continues to be national interest in pursuing spent fuel recycling and advanced nuclear power 
technology.'
\end{solution}

\question Line Items \\
6) 4-63 - What does performance mean in this context?

\begin{solution}
We have added description for the meaning of performance. 

Fuel cycles that involved continuous recycling
of co-extracted U/Pu or U/TRU in fast spectrum critical reactors
consistently scored high on overall performance based on the nine 
DOE-specified evaluation criteria: nuclear waste management, 
financial risk and economics, 
proliferation risk, nuclear material security risk, safety, 
environmental impact, resource utilization, development and deployment 
risk, and institutional issues \cite{wigeland_nuclear_2014}. 
\end{solution}

\question Line Items \\
7) 5-92 - Just curious, what are the CPU demands on using d3ploy?

\begin{solution}
\deploy's CPU demands vary based on the size of the simulation (no. of facilities, etc.) 
and the prediction method used. 
For the complex transition scenarios (`best performance models') discussed in the paper, 
the simulations' CPU demand was very dependent on the prediction method used. It varied from 
10 minutes for the simple \texttt{Moving Average} prediction method to 20 hours for 
\texttt{Stepwise Seasonal}. These simulations were performed on a quad-core Intel Xeon Processor 
E3-1225 V5 work station, not on a supercomputer. 
\end{solution}

\question Line Items \\
8) 5-93 - Do commodities include coolant or reflector materials? Control rods?
\begin{solution}
Commodities do not include coolant, reflector, or control rod materials. The commodity associated 
with each reactor is reactor fuel. However, in \deploy, a user could easily add these other 
materials as commodities and set up a supply chain for each of them.  
\end{solution}

\question Line Items \\
9) 8-Fig 2 - Again, just curiosity, are there plans to include consolidated interim storage or onsite dry
storage in Cyclus? Does the model include outages?

\begin{solution}
A storage facility archetype is available in \Cyclus. The archetype can be customized by the user to act as a 
consolidated interim storage or onsite dry storage. This model does not include outages. 
\end{solution}

\question Line Items \\
10) 11-Sec 2.3 - Could d3ploy be used for hybrid systems; e.g., used with renewables or industrial product?

\begin{solution}
Yes, \deploy can be used for hybrid systems. \deploy is used with any assortment of 
\Cyclus archetypes. However, there are currently no \Cyclus archetypes for modeling hybrid 
systems. An interested user could design \Cyclus archetypes that represent various facilities 
in a hybrid system and then use them with \deploy.
\end{solution}

\question Line Items \\
11) 13-Sec 2.4 - Why were these time series methods selected? (There isn’t any dispute with the selection.)

\begin{solution}
We chose an assortment of non-optimizing, deterministic-optimizing, and stochastic 
optimizing time series methods that are commonly used and readily available in Python 
packages for quick and easy implementation. 
\end{solution}

\question Line Items \\
12) 28-333 cf. Section 3. Why is 6 or 8 extra reactors unrealistic?

\begin{solution}
This passage has been modified to explain why 6 to 8 extra 
reactor unrealistic. It reads: 

We varied the power buffer size for the EG01-EG24 and EG01-EG30 
linearly increasing power demand transition scenarios. 
Figures 10(a), 10(b), 
and Table VI 
show that increasing the buffer size increases the robustness 
of the supply chain by minimizing power undersupply. 
The cumulative undersupply is minimized with a 6000MW and 8000MW 
buffer for EG01-EG24 and EG01-EG30 respectively.
In Figure 10(a), a 4000MW buffer size has 
8 time steps with undersupply, while a 6000MW buffer size has 
7 time steps with undersupply. 
In Figure 10(b), a 2000MW buffer size has 
6 time steps with undersupply, while a 8000MW buffer size has 
5 time steps with undersupply. 
We determined that extra commissioning of multiple reactors does not 
justify a single time step with no undersupply. 
This type of logic is difficult to program into a NFC simulator, 
therefore, even though NFC simulators can help inform policy decisions, 
decision-makers must still evaluate NFC simulator results to determine if 
they are valid and logical. 
Therefore, a buffer of 4000MW and 2000MW minimizes 
the power undersupply for EG01-EG24 and EG01-EG30 transition 
scenarios, respectively.
\end{solution}

\question 
Aside.Why does the manuscript have 2 inch margins? I’m assuming this was prepared in LaTeX, where
a4paper would have been sufficient. This doesn’t have any bearing on the recommendation to publish
the paper, but it just makes it harder to read. I’m actually surprised NT didn’t insist on the standard
format prior to sending it out to reviewers. I think they have a template.

\begin{solution}
I have updated the submission to follow the Nuclear Technology Latex template. 
\end{solution}

\end{questions}

\section*{Reviewer 2}
\begin{questions}

\question Comments on Content.
You make no mention of the impact of the issue of dynamic fuel compositions. 
The commodity that the reactors are demanding and the one being supplied 
by the reprocessing plants is constantly in flux during a transition 
scenario with unlimited recycle. Other applications that use forecasting 
methods don't have this concern.  So effectively what you need to predict 
isn't just the capacities needed based on the mass of SNF, but also its 
post-reprocessing worth. When using reprocessed fuel, the necessary fissile 
loading fraction of MOX or TRU in a fast reactor may be as much as 50\% 
higher for material sourced from a MOX LWR than from a UOX LWR. 
This difference will significantly impact your reprocessing capacity 
required to supply that material.

\begin{solution}
Thank you for your kind review.
This paper focuses on \deploy's capabilities, we have clarified that we use recipe reactors 
and constant reprocessing buffer sizes and isotope efficiencies in this paper on pg 22. It reads: 
`The \texttt{reactor} facility used in the \Cyclus simulation 
is a recipe reactor; it accepts a fresh fuel recipe and outputs 
a spent fuel recipe. 
The recipes used for the LWR, MOX LWR, and 
SFR are based on recipes generated by VISION 
\cite{bae_arfctransition-scenarios_2019}
that closely match EG30 scenario specifications in 
Appendix B of the DOE Evaluation and Screening Study 
(E\&S study) \cite{wigeland_nuclear_2014}. 
The LWR, FR, and MOX LWR cooling pools have a residence time of 36 months, and a maximum 
inventory size of 1e8kg of fuel. 
The reprocessing segment for each reactor type has a reprocessing and mixer facility. 
Each reprocessing facility has a throughput of 1e8kg and separates U/Pu or U/TRU from 
other isotope in spent fuel. 
Each mixer facility mixes the U/Pu or U/TRU to fabricate new reprocessed fuel.  
\deploy will deploy reprocessing facilities based on the demand of reprocessed fuel 
from the MOX LWRs and FRs to ensure that sufficient fissile material 
feeds the reprocessing facilities to make sufficient reprocessed fuel 
for each reactor type. ' 

We agree that for closed NFCs, loops created by the reprocessing plant cause the reactor's 
incoming and outgoing material composition to be dynamic. Therefore, making simple static 
assumptions such as recipe method is a poor approximation 
\cite{bae_neural_2019,peterson-droogh_value_2018}. 
We added this statement to the future work section: 
`The transition scenario simulations in this work assume recipe reactors, 
however, in reality, complexity introduced by the reprocessing plants causes 
each reactor to have dynamic incoming and outgoing material compositions. 
Therefore, the static recipe method assumption is a poor approximation 
\cite{bae_neural_2019,peterson-droogh_value_2018}. 
In future transition scenario work with \deploy, a \Cyclus 
reactor archetype that uses dynamic fuel compositions could be used.  '

\end{solution}

\question 

Pg 2. Fuel cycle options doesn't need to be capitalized. \\
Pg 3. Greenhouse is one word and greenhouse gas doesn't need to be capitalized. \\
Pg 3. Line 50-52. This sentence is hard to follow with nested lists and clunky grammar.\\
Pg 3. Line 59 Evaluation groups doesn't need to be capitalized. \\
Pg 9. This section might be easier to follow if it was an enumerated list rather than forcing a paragraph structure.\\
Pg 9. You use the acronym LWR but don't define it until page 11.\\
Pg 10. Please use the same indentation for both equations 4 and 5.\\
Pg 12. Numbers on axes for figure 3 are small.\\
Pg 13. The definition of terms in an equation shouldn't have its own equation number. Equation 7 should be a list or part of equation 6. Also, please match indentation.\\
Pg 15. Please fix indentation after equation 9.\\
Pg 15. The sentence after an equation does not always start a new paragraph, so it shouldn't be indented if it doesn't.\\
Pg 16. Equation 12 should be a list or part of equation 11. \\
Pg 16. You do not define what the "L" term is in equation 13.\\
Pg 17. You do not define what the "d" or "Y" terms are in equation 14.\\
Pg 18. Rather than starting the results section with a sentence 7 lines long, it would improve readability to make it into an enumerated list.\\
Pg 19. You don't need to state in the caption to figure 5 that power undersupply is avoided. You state that in the preceding paragraph
Pg 23. The labels and axes are too small in figure 8. You also don't need to state results in your caption that are already stated in the main text.\\
Pg 24. The labels and axes are too small in figure 9. You also don't need to state results in your caption that are already stated in the main text.\\
Pg 25. Why is table 7 placed on page 31, if it's only mention is here and before table 5 and 6 are referenced?\\
Pg 27. The font size used for figure 10a and b is good, but does not match the rest of your figures. All figure titles and labels should use the same font and font size. \\
Pg 28. Please provide a citation for "The need for commodity supply buffers is a reflection of reality in which a supply buffer is usually maintained to ensure continuity in the event of an unexpected failure in the supply chain." \\
Pg 29. The axes' font is too small in figure 11 and 12.\\
\begin{solution}
All these comments have been addressed and fixed. 
\end{solution}

\question
Pg 5. It is unclear if d3ploy runs before the simulation or if it is 
doing these calculation on-the-fly. If you are only predicting the 
necessary capacity a single time-step in advance does that mean that 
there is no deployment or process times?

\begin{solution}
\deploy does these calculations on-the-fly. Yes it is assumed that deployment occurs 
within one time step (month). The passage on Pg 5 has been edited to clarify these 
questions: 

During a \Cyclus simulation, at every time step, \deploy 
predicts the supply and demand of each commodity for the next time 
step. 
It is assumed that facility deployment occurs 
within one time step (month). 
\end{solution}

\question
Pg 24. In section 3.2 you state what methods provide the best results, but why do the 
POLY and FFT methods perform best when they do? What is it about those methods that 
causes them to outperform the others so significantly?

\begin{solution}
A good question. The following passage on pg 27 has been enhanced to explain 
why the \texttt{POLY} and \texttt{FFT} methods outperform the others: 

Figures 8, 9, and Table 
IV show that the \texttt{POLY} method 
minimizes power undersupply for constant power transition scenarios, 
and the \texttt{FFT} method minimizes power undersupply for linearly increasing 
power transition scenarios. 
Undersupply and under-capacity of commodities occur during two main time periods: 
initial demand for the commodity and during the transition period (month 950 onwards).
The \texttt{POLY} method minimizes commodity undersupply during the transition period, 
and does especially well during the start of the simulation in Figure 8.  
We hypothesize that it is because a first order polynomial was used, and thus, \texttt{POLY}
could best predict the future demand of each commodity. 
The \texttt{FFT} method struggled with predicting the demand at the start of the simulation
in both Figures 8 and 9, 
but did very well during the transition period for both simulations. 
The reason why it is so effective is that is able to capture the significant features 
of the time series data and uses it to predict future demand values. 
It is weaker at the beginning of the simulation because there is a lack of time series data. 
Different methods perform well for different power demand curves. In 
\cite{chee_demonstration_2019}, we demonstrate that the \texttt{HOLT-WINTERS} method 
minimizes undersupply of all commodities for a sinusoidal power demand curve. 
This is because the triple exponential smoothing
method excels in forecasting data points for repetitive seasonal
series of data \cite{chee_demonstration_2019}.
\end{solution}

\question 
Pg 29. Figure 11b and 12b shows that facilities remain in the simulation after 
they are created regardless of time and need. Is this the expected behavior, 
and if it is, is there no facility decommissioning?

\begin{solution}
This is not the expected behavior, however, we ran these simulations before the 
decommissioning option was developed in \deploy. We deemed the no decommissioning assumption 
acceptable at the time, since nuclear fuel cycle facilities historically have not been 
decommissioned in a timely fashion after need for them stops. 
The Nuclear Regulatory Commission states that a nuclear site must be decommissioned 
within 60 years of plant ceasing operations \cite{noauthor_portsmouth_nodate}. 
Now, \deploy \cite{chee_arfc/d3ploy:_2019} 
has decommissioning capabilities, 
and future transition scenario analyses using \deploy have the option to use it.  
The user defines the number of time steps with an oversupply of a commodity 
to trigger the decommissioning of the facility that produces it. 
\end{solution}

\question
Pg 29. Please include descriptions of your facilities, or at least their capacities. 
Showing the number of agents deployed in figure 11 and 12 doesn't have much meaning 
otherwise (e.g., it could be misinterpreted in figure 11 that power supplied is 
increasing since you are deploying significantly more reactors).

\begin{solution}
We have added the following passage on pg 22 to address this concern: 

The facilities used in the transition scenario simulations are described below. 
The source facility has a throughput of 1e8kg of natural uranium, and the enrichment 
facility has a SWU capacity of 1e100. The LWRs have an 
assembly size of 29863.3kg with 3 assemblies per core, and a power capacity of 1000 MW. 
The FRs have an assembly size  of 3950kg and power capacity of 333.34 MW.
The MOX LWRs have an assembly size of 33130kg and power capacity of 1000 MW. 
The \texttt{reactor} facility used in the \Cyclus simulation 
is a recipe reactor; it accepts a fresh fuel recipe and outputs 
a spent fuel recipe. 
The recipes used for the LWR, MOX LWR, and 
SFR are based on recipes generated by VISION 
\cite{bae_arfctransition-scenarios_2019}
that closely match EG30 scenario specifications in 
Appendix B of the DOE Evaluation and Screening Study 
(E\&S study) \cite{wigeland_nuclear_2014}. 
The LWR, FR, and MOX LWR cooling pools have a residence time of 36 months, and a max 
inventory size of 1e8kg of fuel. 
The reprocessing segment for each reactor type has a reprocessing and mixer facility. 
Each reprocessing facility has a throughput of 1e8kg and separates U/Pu or U/TRU from 
other isotope in spent fuel. 
Each mixer facility mixes the U/Pu or U/TRU to fabricate new reprocessed fuel.  
\deploy will deploy reprocessing facilities based on the demand of reprocessed fuel 
from the MOX LWRs and FRs to ensure that sufficient fissile material 
feeds the reprocessing facilities to make sufficient reprocessed fuel 
for each reactor type. 
Each waste repository is assumed to have infinite capacity. 
For more details about each simulation, the input files can be found at 
\cite{bae_arfctransition-scenarios_2019}. 
\end{solution}

\end{questions}
\bibliographystyle{unsrt}
\bibliography{../2019-d3ploy-transition.bib}
  \end{document}

  %
  % End of file `elsarticle-template-num.tex'.
