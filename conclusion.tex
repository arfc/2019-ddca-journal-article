\section{Conclusion}
The present nuclear fuel cycle in the United States is a once-through 
fuel cycle of LWRs with no used fuel reprocessing. 
This nuclear fuel cycle faces cost, safety, proliferation, and spent 
nuclear fuel challenges that hinder large-scale nuclear power deployment. 
The U.S Department of Energy identified future 
nuclear fuel cycles, involving continuous recycling of co-extracted U/Pu or 
U/TRU in fast and thermal spectrum reactors, that may overcome these challenges.
These transition scenarios have been modeled previously in the following 
nuclear fuel cycle simulators \cite{feng_standardized_2016,bae_standardized_2019}: 
ORION, DYMOND, VISION, MARKAL, and \Cyclus. 
However, for many nuclear fuel cycle simulators, the user is required to 
define a deployment scheme for all supporting facilities to avoid any 
supply chain gaps or resulting idle reactor capacity. 
Manually determining a deployment scheme for a once-through 
fuel cycle is straightforward; however, for complex fuel cycle 
scenarios, it is not. 
In this paper, we introduce the capability, \deploy, in \Cyclus 
that automatically deploys 
fuel cycle facilities to meet user-defined power demand.     
In this paper, we demonstrate that with careful selection of 
\deploy parameters, we can 
completely automate the setup of constant and 
linearly increasing power demand transition scenarios
for EG01-23, EG01-24, EG01-29, and EG01-30 with minimal 
power undersupply. 
Using \deploy to set up transition scenarios 
saves the user simulation set-up time, making it more efficient 
than the previous efforts that
required a user to manually calculate and use trial and error 
to set up the deployment scheme for the supporting fuel cycle 
facilities. 
Transition scenario simulations set up manually are sensitive 
to changes in the input parameters resulting in an 
arduous setup process since a slight change in one 
input parameter would result in the need to recalculate 
the deployment scheme to ensure no undersupply 
of power.   
Therefore, by automating this process, when the user varies input parameters 
in the simulation, \deploy automatically adjusts the
deployment scheme to meet the new constraints. 