\section{Conclusion}
In this paper, we demonstrate that by carefully selecting 
\deploy parameters, we are able to 
effectively automate setting up of constant and 
linearly increasing power demand transition scenarios
for EG01-23, EG01-24, EG01-29, and EG01-30 with minimal 
power undersupply. 
Using \deploy to set up transition scenarios 
is more efficient than the previous efforts that
required a user to manually calculate and use trial and error 
to set up the deployment scheme for the supporting fuel cycle 
facilities. 
Transition scenario simulations set up this way are sensitive 
to changes in the input parameters resulting in an 
arduous setting up process, since a slight change in one 
of the input parameters would result in the need to recalculate 
the deployment scheme to ensure that there is no undersupply 
of power.   
Therefore, by automating this process, the user can vary input parameters 
in the simulation and \deploy will automatically adjust the
deployment scheme to meet the new constraints. 