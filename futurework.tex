\section{Future Work}
We simulate transition scenarios to predict the future; 
however, when implemented in the real world, the transition 
scenario tends to deviate from the optimal scenario.
Therefore, nuclear fuel cycle simulators must be used to conduct
sensitivity analysis studies to understand the subtleties of 
a transition scenario better to reliably inform policy decisions.
Previously it was difficult to conduct sensitivity analysis with \Cyclus 
as users have to manually calculate the deployment scheme for a 
single change in an input parameter. 
By using the \deploy capability,
sensitivity analysis studies are more efficiently 
conducted as facility deployment in transition scenarios 
are automatically set up. 
\deploy will also be open-source and available for the forseeable future on github 
\cite{chee_arfc/d3ploy:_2019}, to be used with \Cyclus for conducting any 
transition scenario analysis. 
The transition scenario simulations in this work assume recipe reactors, 
however, in reality, complexity introduced by the reprocessing plants causes 
each reactor to have dynamic incoming and outgoing material compositions. 
Therefore, the static recipe method assumption is a poor approximation 
\cite{bae_neural_2019,peterson-droogh_value_2018}. 
In future transition scenario work with \deploy, a \Cyclus 
reactor archetype that uses dynamic fuel compositions could be used. 
